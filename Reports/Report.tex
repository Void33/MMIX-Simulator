\documentclass[a4paper,11pt]{report}
\usepackage{hyperref}
\usepackage{appendix}
\renewcommand{\bibname}{References}
\begin{document}
\begin{titlepage}
	\centering
	{\bfseries Component Based MMIX Simulator using Multiple Programming Paradigms \par}
	\vspace{1cm}
	{A dissertation submitted in partial fulfilment of the requirements for the MSc in Advanced Computing Technologies\par}
	\vspace{1.5cm}
	{by Stephen Edmans\par}
	\vspace{2cm}
	{Department of Computer Science and Information Systems\par}
	{Birkbeck College, University of London\par}
	\vspace{2cm}
	{\large September 2015\par}
\end{titlepage}
\newpage
\chapter*{} % Academic Declaration
{This report is substantially the result of my own work except where explicitly
indicated in the text. I give my permission for it to be submitted to the JISC
Plagiarism Detection Service. I have read and understood the sections on plagiarism
in the Programme Handbook and the College website.\par}
\vspace{1cm}
{\noindent The report may be freely copied and distributed provided the source is explicitly
acknowledged.}
\chapter*{Abstract}
\addcontentsline{toc}{chapter}{Abstract}
MMIX was first proposed by Donald E. Knuth in \cite{knuth:aocp1}
\newpage
\addcontentsline{toc}{chapter}{Contents}
\tableofcontents
\newpage
\listoffigures
\newpage
\chapter*{Acknowledgements}
\addcontentsline{toc}{chapter}{Acknowledgements}
\chapter{Introduction}
\chapter{Assembler}
\section{Introduction}
\section{Executable}
%% HOW WE RUN THE ASSEMBLER, WHAT THE PARAMETERS ARE & WHAT THE OUTPUT IS
\section{Lexer}
\section{Parser}
\section{Code Generation}
\subsection{Symbol Table}
\subsection{Automatically Assigned Registers}
\subsection{Local Symbols}
%What is a local symbol
%How do we achieve this?
%Convert all #H statements with system generated labels that cannot be used by the user (??LS#H*) Where # is the local label number and * is a counter
%Unclear what to do if local symbol defined on line where it could be used.
\subsection{Handling Operands}
\subsection{Assembler Directives}
\subsection{Generating the Output}
\section{Component Testing}
%% DO I NEED A SECTION HERE TO TALK ABOUT ANY ISSUES AND THOUGHTS RESULTING IN THE DEVELOPMENT, ALONG WITH THE INTEGRATION OF THE PARTS
\chapter{Graphical User Interface}
\section{Introduction}
\section{User Interface Design}
\subsection{Console Panel}
\subsection{Controls Panel}
\subsection{Main State Panel}
\subsection{Memory Panel}
\subsection{Registers Panel}
\section[Asynchronous UI Programming with Actors]{Asynchronous User Interface Programming with Actors}
\section{Communication}
\section{Component Testing}
\chapter{Virtual Machine}
\section{Introduction}
\section{Memory}
\section{Registers}
\section{Central Processing Unit}
\section{Calling the Operating System}
\section{Communication}
\section{Component Testing}
\chapter{Simulator Application}
\section{Introduction}
%% HOW THE APPLICATION AS A WHOLE WORKS
\section{Integration Testing}
\subsection{Generate Prime Numbers Sample Application}
\chapter*{Conclusion}
\addcontentsline{toc}{chapter}{Conclusion}
\nocite{*}
\bibliographystyle{acm}
\bibliography{bibtex}
\addcontentsline{toc}{chapter}{References}
\begin{appendices}
\noappendicestocpagenum
\addappheadtotoc 
\chapter{Source Code}
\section{Assembler}
\section{Graphical User Interface}
\section{Virtual Machine}
\end{appendices}
\end{document}

